\documentclass[a4paper,11pt]{article}

\usepackage[utf8]{inputenc}
\usepackage[francais]{babel}
\usepackage{amsmath}

%
\title{Survey: Lattice Reduction techniques to attack RSA}
\author{David Wong}
\date{March 2015}


\begin{document}

\maketitle

\begin{abstract}
\textbf{RSA}, carrying the names of \textbf{Ron Rivest}, \textbf{Adi Shamir} and \textbf{Leonard Adleman}, is one of the first practicable \textbf{public-key} cryptosystem. The algorithm was publicly described for the first time in \textbf{1977} and has been since the most used cryptosystem when it comes to asymmetric problems. For now more than \textbf{30 years}, Cryptanalists and Researchers have looked for ways to \textbf{attack RSA}.\\
In 1995, \textbf{Coppersmith} released a paper on how to attack RSA using \textbf{Lattices} and \textbf{Lattice reduction techniques} such as \textbf{LLL}. A few years later, \textbf{Howgrave-Graham} revisited Coppersmith's algorithm and made it easier to understand and apply.\\
Attacks based on Lattice reduction techniques caught up and several researches were done on the subject. Years after \textbf{Wiener} found that you could successfully break RSA if the private exponent was too small. In 2000, \textbf{Boneh} and \textbf{Durfee} found a better attack, still based on Lattice, that was simplified afterwards by a work from \textbf{Herrmann} and \textbf{Mayers}.\\
In this survey we will see in what \textbf{model} the attacks are taking place. We will explain 

\end{abstract}

\newpage



\newpage

\section{RSA}\label{rsa}

Let's quickly recall how RSA works.

\section{Attacks}\label{attacks}

\end{document}